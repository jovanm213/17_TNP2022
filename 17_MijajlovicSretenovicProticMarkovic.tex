% !TEX encoding = UTF-8 Unicode

\documentclass[a4paper]{article}

\usepackage{color}
\usepackage{url}
\usepackage[T2A]{fontenc} % enable Cyrillic fonts
\usepackage[utf8]{inputenc} % make weird characters work
\usepackage{graphicx}

\usepackage[english,serbian]{babel}
%\usepackage[english,serbianc]{babel} %ukljuciti babel sa ovim opcijama, umesto gornjim, ukoliko se koristi cirilica

\usepackage[unicode]{hyperref}
\hypersetup{colorlinks,citecolor=green,filecolor=green,linkcolor=blue,urlcolor=blue}

\DeclareUnicodeCharacter{0301}{\'{e}}

%\newtheorem{primer}{Пример}[section] %ćirilični primer
\newtheorem{primer}{Primer}[section]

\title{Roboti kao nastavnici\small \\Seminarski rad u okviru kursa\\Tehničko i naučno pisanje\\Matematički fakultet}
\author{Jovan Mijajlović, jovan.mijajlovic03@gmail.com\\Miona Sretenović, sretenovicmiona7@gmail.com\\Mina Protić, minaproticc@gmail.com\\Mihailo Marković, mihailoo003@gmail.com}
\date{7. decembar 2022.}

\begin{document}
\maketitle

\abstract{
Tema ovog seminarskog rada su edukativni roboti i njihova primena u nastavi. Postavlja se glavno pitanje, da li roboti mogu da zamene ljude u nastavi? U nastavku rada potrudićemo se da odgovorimo na ova pitanje.
}

\tableofcontents
\newpage

\section{Uvod}
\label{sec:uvod}
Razvoj tehnologije kao i razvoj računara i njihovih znanja poslednjih decenija veoma brzo je napredovao što je omogućilo veću primenu informacionih tehnologija, koje su dosta uticale na naš svakodnevni život. Zbog sve veće upotrebe digitalnih tehnologija u svakodnevnom životu javlja se sve veća potreba za digitalnim opismenjavanjem ljudi. Kako stariji tako i mladi danas moraju imati neko “osnovno” znanje kada je reč o informacionim tehnologijama.
U nastavku obradićemo teme: inovacije i implementacija tehnologije u nastavi i obrazovanju, nastavnicima, njihovoj ulozi i uticaju na nastavu i robotima u ulozi nastavnika.

\section{Implementacija tehnologije u nastavi}
\label{sec:naslov1}

Informacione tehnologije su postale deo naših života, navikli smo da koristimo računare, telefone, pametne satove, koji dosta olakšavaju naš život. Stoga potrebno je da svi znamo da ih koristimo. Na primer, danas nije potrebno ići do pošte ili banke, kako bismo platiti račune, već je sve moguće uraditi pomoću telefona ili računara, u školama se više ne koristi stari “tradicionalni” dnevnik, nego postoji elektronski dnevnik, kao i elektronski indeks na fakultetima.

Za vreme pandemije korona virusa u skoro svim školama bila je uvedena onlajn nastava, koja predstavlja proces edukacije putem interneta. Tokom onlajn nastave koristili su se razni internet servisi, kao što su video pozivi, digitalni resursi, poput prezentacija, video snimaka, audio predavanja i pdf vodiča koje nam je omogućila najnovija tehnologija... (slika \ref{fig:online}) U skoro svim osnovnim školama informatika je postala obavezan predmet, na kome đaci od malena uče kako da koriste savremenu tehnologiju. U većini srednjih škola se uče razni programski jezici i programiranje. Svi uče i znaju kako da koriste osnovne servise, kao što je Microsoft Office paket.

\begin{figure}[ht!]
\begin{center}
\includegraphics[scale=0.23]{online.jpg}
\end{center}
\caption{Onlajn nastava}
\label{fig:online}
\end{figure}

\newpage
\section{Nastavnici - uloga i uticaj na nastavu}
\label{sec:naslov2}

Nastavnik je stručna osoba visokih radnih, obrazovnih i etičkih kvaliteta edukovana za rad u vrtićima, školama ili fakultetima za određen predmet.
Nastavnici koji pokazuju entuzijazam prema materijalima kursa i studentima mogu stvoriti bolju atmosferu za učenje. Postoji velika verovatnoća da će njihovi učenici biti angažovaniji, zainteresovaniji i energičniji za rad. (slika \ref{fig:nastavnik})

\begin{figure}[ht!]
\begin{center}
\includegraphics[scale=1.3]{nastavnik.jpg}
\end{center}
\caption{Nastavnik}
\label{fig:nastavnik}
\end{figure}

Motivacija učenika za učenje dovodi se u vezu sa njihovim odnosom sa nastavnikom. Nastavnici entuzijasti su posebno dobri u stvaranju korisnih odnosa sa svojim učenicima. Učenici koji ostvare bolji odnos sa svojim nastavnikom postižu veći lični i akademski uspeh.

Stvari koje se traže od nastavnika:
\begin{itemize}
\item{} Znanje - kako o samom predmetu tako i znanje o tome kako ga predavati.
\item{} Zanatske veštine - planiranje časa, korišćenje nastavnih tehnologija, upravljanje učenicima i grupama, praćenje i procena učenja.
\item{} Dispozicije - suštinske vrednosti i stavovi, uverenja i posvećenost.
\end{itemize}


\newpage
\section{Roboti}
\label{sec:naslov3}

Robot je elektro-mehanička jedinica koja je u stanju da po nekom programu, ili pod kontrolom čoveka izvodi određene zadatke. Inteligenciju koju robot poseduje čini program, koji određuje sposobnost robota da prepozna određene situacije i da se u njima ponaša na određeni način ili da čak iz sopstvenog iskustva uči kako da se snalazi u novim situacijama i rešava nove probleme. Ova vrsta inteligencije se još i zove veštačka inteligencija i predstavlja zasebnu granu nauke. (slika \ref{fig:robot})

\begin{figure}[ht!]
\begin{center}
\includegraphics[scale=0.15]{robot1.jpg}
\end{center}
\caption{Robot}
\label{fig:robot}
\end{figure}

Roboti koji imaju oblik ljudskog tela zovu se humanoidni roboti. Ako roboti imaju karakteristike, poput kretanja, govora, gestikulacije, radi se o androidima. Ovaj termin se ipak češće sreće u naučnoj fantastici. 

U tabeli možete pogledati opis nekih humanoidnih i androindnih robota. (tabela \ref{tab:tabela1})

\begin{table}[ht!]
\begin{center}
\caption{Humanoidni prijatelji}
\begin{tabular}{|c|c|} \hline
Ime robota& Opis robota\\ \hline
Bina48 & Robot student i predavač\\ \hline
Osmo Awibe &Uz njega deca razvijaju motoričke veštine\\ \hline
Ultimate 2.0 &Napredni programabilni robotski komplet\\ \hline
Miro &Baziran na veštačkoj inteligenciji i senzorima\\ \hline
mBot &STEM robot za kodiranje\\ \hline
Jimu &Zahvaljujući njemu učenici razvijaju ljubav prema robotima\\ \hline
\end{tabular}
\label{tab:tabela1}
\end{center}
\end{table}

\newpage
\subsection{Bina48}
\label{subsec:podnaslov1}

Bina48 je jedan od najnaprednijih društvenih robota na svetu. Inspirisana je ženom po imenu Bina Rothblatt, koja je udata za tehnološkog preduzetnika Martina Rothblatta. Robot je imao veliku medijsku pažnju od kada je stvoren 2010. godine i više puta je spominjan kao „najsenzacionalniji robot na svetu“. (slika \ref{fig:bina48})

Bina48 koristi veštačku inteligenciju zasnovanu na sećanjima, stavovima, uverenjima i ponašanju Bine Rothblatt za interakciju sa ljudima. Osim toga, njeno pamćenje može biti ispunjeno znanjem sa interneta ili drugim informacijama kojima je "hranjena". Ona ima pokretno lice, oči koje imaju sposobnost vida, uši koje čuju i digitalnu memoriju koja omogućava razgovar sa njom. Naprednog robota kreirala je kompanija Hanson Robotics. Bina48 je deo projekta LifeNaut, eksperimenta veštačke inteligencije i „sajber svesti“.

Bila je student pre nego što je postala nastavnik. Bila je deo predavanja u učionici Vilijama Berija, profesora filozofije. Zajedno sa Brusom Dankanom, direktorom Fondacije Terasem, složio se da puste Binu48 da razgovara sa njegovim studentima. Jednog dana kada je Bina48 pohađala Berijevu nastavu, izrazila je želju da pohađa koledž. Nedugo zatim, Bina48 je dobila 16-nedeljnu obuku iz ljubavne filozofije na Univerzitetu Notre Dam de Namur u Belmontu.

Postala je prvi robot koji je predavao kurs na nivou koledža. Zajedno sa profesorom Barijem i docentom majorom Skotom Parsonsom, Bina48 je predavala dva dela uvodne obuke iz filozofije etike. Obuka je pokrivala teorije rata i upotrebe veštačke inteligencije u društvu. 
 
\begin{figure}[ht!]
\begin{center}
\includegraphics[scale=0.3]{Bina48.jpg}
\end{center}
\caption{Bina48}
\label{fig:bina48}
\end{figure}

\newpage
\subsection{Primena robota u nastavi}
\label{subsec:podnaslov2}

Roboti u nastavi mogu imati pasivnu ulogu, koriste se kao alat za učenje ili kao nastavno sredstvo. Dok sa druge strane mogu imati i aktivnu ulogu vršnjaka i prijatelja. Primer robota vršnjaka je ASIMO robot koji podstiče i motiviše ostale učenike da se opredele za naučne i inženjerske oblasti. Takođe postoji još jedan način na koji roboti mogu učestvovati u obrazovanju, a to je da budu nastavnici. \\
U učenju stranih jezika, alat za učenje pomaže učeniku da savlada određene fraze kroz interakciju ili igricu, dok robot vršnjak za izgovorenu reč vraća povratnu informaciju, da li je dobro izgovorena, dok se robot nastavnik trudi da pomogne učenicima u pamćenju novih reči i širenju vokabulara. 
U savladavanju matematičkih zadataka robot nastavnik prilagođava težinu aritmetičkih izraza u odnosu na učenikovo znanje. Robot vršnjak zajedno sa učenikom rešava zadatke davajući učeniku nagoveštaje kada mu je to potrebno.

\section{Zaključak}
\label{sec:zakljucak}

Na osnovu navedenog, možemo zaključiti da je uključenje robota u nastavu korisno, možda čak i neizbežno, zbog sve većeg razvoja savremenih tehnologija. Ipak oni neće moći u potpunosti da zamene čoveka kao nastavnika, pre svega zbog nemogućnosti da ostvare kontakt sa učenicima, kao što to rade ljudi. Takodje, emocionalna interakcija sa učenicima nikada neće moći da bude zamenjena veštačkom inteligencijom. Ukoliko želimo da ovaj sistem rada ima što bolju primenu, neophodno je da obrazovni sistem investira u istraživanja i razvoj sredstava koji bi omogućili kako učenicima, tako i profesorima bolje okruženje za rad.

\newpage
\addcontentsline{toc}{section}{Literatura}
\appendix

\iffalse
\bibliography{seminarski} 
\bibliographystyle{plain}
\fi

\begin{thebibliography}{9}

\bibitem{}Wikipedia, https://sr.wikipedia.org/wiki/Робот

\bibitem{}https://www.rts.rs/page/magazine/sr/story/1882/tehnologija/2642727/roboti-stizu-u-skolske-klupe.html

\bibitem{}https://learnenglish.britishcouncil.org/skills/reading/b1-reading/robot-teachers

\bibitem{}https://www.festo.com/rs/sr/e/casopis/gospodin-i-gospoda-roboti-kao-nastavnici-id\_35076/

\bibitem{}https://www.oberlo.com/blog/online-teaching

\bibitem{}https://frontcore.com/blog/meet-the-first-robot-who-teaches-a-college-course/

\bibitem{Onlajn nastava}Dostupno na adresi: 
https://www.forbes.com/sites/enriquedans/2020/11/30/whether-we-like-it-or-not-online-teaching-is-the-future-so-lets-start-learning-how-to-do-itproperly/?sh=6532c2bd53ff

\bibitem{Nastavnik}Dostupno na adresi: https://www.biznisipravo.rs/kako-se-zaposljavaju-nastavnici-u-skolama/

\bibitem{Robot}Dostupno na adresi: https://www.amazon.com/Ruko-Programmable-Interactive-Control-Present/dp/B085WPHTHW

\bibitem{Bina48} Dostupno na adresi: https://frontcore.com/blog/meet-the-first-robot-who-teaches-a-college-course/

\bibitem{Tabela}Tabela napravljena uz pomoć sajta: https://www.savremena-osnovna.edu.rs/robot-pepper-stvarno-drugaciji-nastavnik-u-savremenoj/

\end{thebibliography}

\end{document}
