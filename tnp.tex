\documentclass[]{article}

\title{Roboti kao nastavnici \\ Seminarski rad u okviru kursa \\ Tehničko i naučno pisanje \\ Matematički fakultet}
\author{ Miona Sretenović \\ sretenovicmiona7@gmail.com }
\date{Octobar 2022}

\begin{document}
\maketitle
\tableofcontents

\section{Uvod} 
Veštačka inteligencija nas u svojstvu pomagača sve češće prati kroz svakodnevni život - robota ima čak i u učionicama. Počelo je doba pravih kognitivnih sistema. Prošla su vremena kada su istraživači u računare programirali statično znanje. Umesto toga, danas ljudi rade sa metodama koje robotima i drugim mašinama omogućavaju aktivno učenje, primenu i stavljanje naučenog u sve veće kontekste. Robot ponekad postane i učitelj. Današnja veštačka inteligencija dizajnirana je tako da uči iz iskustva. Ovo nagomilano bogatstvo znanja je od velike vrednosti za ljude. To je jedan od razloga zašto se roboti danas sve više koriste u obrazovanju. Zbog svojih komunikacionih sposobnosti koje se kontinuirano razvijaju, sve češće prenose specijalizovana znanja u školama i na univerzitetima ili kao privatni učitelji.
\section{Literatura}
\end{document}
